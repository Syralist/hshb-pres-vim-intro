\documentclass[aspectratio=1610,svgnames]{beamer}

\usepackage{lmodern}
\usepackage[T1]{fontenc}
\usepackage[ngerman]{babel}
\usepackage{selinput}
\SelectInputMappings{%
   adieresis={ä},
   germandbls={ß}
   }

\usetheme{PaloAlto}  %% Themenwahl
\usepackage{csquotes}

\setbeamercovered{transparent}
%\setbeamertemplate{footline}[frame number]
\usecolortheme{spruce}		% grün
\usecolortheme[named=MSUgreen]{structure}
\newcommand{\hshblogo}{\includegraphics[width=1.3cm]{space-logo}}
\newcommand{\divider}[1]{\begin{frame} %
\begin{alertblock}{} %
\centering\usebeamerfont{section title}#1 %
\end{alertblock} %
\end{frame}}
 
\title{vim --- Eine Einführung}
\author{Thomas Helmke}
\date{11.10.2016}
\logo{\includegraphics[width=1.1cm]{space-logo}}
 
\begin{document}
\maketitle
\frame{\tableofcontents}

\section{Einleitung}
\divider{\insertsection}
\begin{frame}[<+->] %%Eine Folie
	\frametitle{Warum vim?} %%Folientitel
	\begin{itemize}
		\item Warum nicht?
        \item Effizientes Arbeiten (nach der Lernkurve)
		\item Ohne Maus bedienbar
	\end{itemize}
\end{frame}
\begin{frame}[<+->] %%Eine Folie
    \frametitle{Unterschied zu \enquote{normalen} Editoren} %%Folientitel
	\begin{itemize}
		\item Verschiedene Modi für unterschiedliche Zwecke
		\item Eigener Kommandosatz
		\item Für reine Tastaturbedienung konzipiert
	\end{itemize}
\end{frame}

\section[Die Modi]{Die verschiedenen Modi}
\divider{\insertsection}
\begin{frame}[<+->]
    \frametitle{Normal-Modus}
    \begin{itemize}
        \item Der Grundmodus
        \item Zum Navigieren durch die Datei
        \item Mit <ESC> kommt man zurück
    \end{itemize}
\end{frame} 
\begin{frame}[<+->]
    \frametitle{Visual-Modus}
    \begin{itemize}
        \item Zum Text markieren
        \item Zeichenweise, Zeilenweise, Blockweise
        \item Hier funktioniert die Maus
    \end{itemize}
\end{frame} 
\begin{frame}[<+->]
    \frametitle{Insert-Modus}
    \begin{itemize}
        \item Erreichbar durch i, a, o, \ldots
        \item Hier kann man endlich Text eingeben
        \item Nutzung wird geringer je mehr man von vim lernt
    \end{itemize}
\end{frame} 

\section{Der Einstieg}
\divider{\insertsection}
\begin{frame}[<+->]
    \frametitle{Die ersten Befehle}
    \begin{description}
        \item[<ESC>] zurück in den Normal-Modus
        \item[i] in den Insert-Modus wechseln, Cursor steht vor dem aktuellen Zeichen
        \item[a] in den Insert-Modus wechseln, Cursor steht hinter dem aktuellen Zeichen
        \item[cw] löscht bis zum Wortende und wechselt in den Insert-Modus
    \end{description}
\end{frame}
\begin{frame}[<+->]
    \frametitle{Weitere Befehle}
    \begin{description}
        \item[o] neue Zeile unter dem Cursor einfügen und in den Insert-Modus wechseln
        \item[u] letzte Änderung rückgängig machen
        \item[.] letzte Aktion wiederholen
        \item[/] Suchen
    \end{description}
\end{frame}
\begin{frame}[<+->]
    \frametitle{Copy \& Paste}
    \begin{description}
        \item[yw] bis zum Wortende kopieren
        \item[yy] ganze Zeile kopieren
        \item[dw] bis zum Wortende ausschneiden
        \item[p] hinter (unter) dem Cursor einfügen
    \end{description}
\end{frame}
\begin{frame}[<+->]
    \frametitle{Datei speichern und schließen}
    \begin{description}
        \item[:w] Datei speichern
        \item[:q] vim beenden
        \item[:q!] vim beenden, ohne zu speichern
        \item[:wq] speichern und beenden
    \end{description}
\end{frame}

\section[Bewegung]{Bewegungsbefehle}
\divider{\insertsection}
\begin{frame}[<+->]
    \frametitle{Innerhalb einer Zeile}
    \begin{description}
        \item[w] zum nächsten Wortanfang
        \item[b] zum vorherigen Wortanfang
        \item[0] Anfang der Zeile
        \item[\$] Ende der Zeile
    \end{description}
\end{frame}
\begin{frame}[<+->]
    \frametitle{Ausserhalb der Zeile}
    \begin{description}
        \item[gg] erste Zeile
        \item[G] letzte Zeile
        \item[\{] Absatzanfang
        \item[t] zum nächsten eingegebenen Zeichen
    \end{description}
\end{frame}

\section{Weitere Infos}
\divider{\insertsection}
\begin{frame}
    \frametitle{Weitere Infos}
    \begin{itemize}
        \item \url{https://github.com/Syralist/hshb-pres-vim-intro}
        \item \url{http://benmccormick.org/2014/06/30/learning-vim-in-2014-the-basics/}
        \item \url{http://yehudakatz.com/2010/07/29/everyone-who-tried-to-convince-me-to-use-vim-was-wrong/}
        \item \url{http://vim-adventures.com/}
    \end{itemize}
\end{frame}
\end{document}
